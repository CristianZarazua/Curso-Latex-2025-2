% Como empezar un documento,

\documentclass{article} % Tipo de documento (Artículo) también puede ser book o beamer.

% Inicio del preámbulo,

% Para insertar paquetes se utiliza el comando:

\usepackage{graphicx} % 'graphicx' es un paquete para insertar imágenes,
\usepackage{lipsum} % 'lipsum' es un paquete para insertar texto de relleno,
\usepackage{amsmath} % 'amsmath' es un paquete para insertar fórmulas matemáticas,

% También pueden insertar macros, pero estas se deben configurar antes,

% Fin del preámbulo

\begin{document} % Inicio del cuerpo del documento,

% Aquí comienza tu contenido que quieres visualizar en el documento, ya sean imágenes, texto, tablas, etc,

\lipsum[1-2] % Inserta texto de relleno,

% También puedes añadir secciones, subsecciones, subsubsecciones,

\section{Ejemplo 1}

% Igualmente operaciones matemáticas, esto con el paquete 'amsmath',

\[\int_{a}^{b} f(x) = F(x)\]

\end{document} % Fin del cuerpo del documento.
